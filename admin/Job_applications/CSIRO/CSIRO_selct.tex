\documentclass[12pt,a4paper]{article}

\usepackage{times}

\textheight=25cm
%\textwidth=16cm 
%\topmargin=-4cm
\oddsidemargin=0cm

\pagestyle{myheadings}
\markright{OMN - 1999\ \hrulefill\ }

\setcounter{page}{5}

\begin{document}

\section*{Re:ST99--11. Response to selection criteria}

\subsection*{A) Problem solving}
\begin{enumerate} 
  \item Problem solving in computer programming has been an integral part
  of all my activities during my education as well as afterwards.
  Examples are development of a public-key crypto system based on the 
  RSA algorithm written in SIMULA, 
  a strongly object-oriented simulation package for 
  dynamic systems mixed with discrete events written in Turbo Pascal,
  a toolbox for wavelet based data analysis written in Matlab/MEX, 
  and development of very fast wavelet transforms written in Fortran.
  \item In my current position I do research in Data Mining with access
  to real data from NRMA, ACTEW, and Mount Stromlo Observatory.
  These data sets are very large, often with missing entries and noise,
  so they must be cleaned in various ways. We are expecially concerned
  with smoothing in high dimensions in order to provide a model of the 
  data that can be treated further with for example 
  clustering algorithms, classifiers or decision trees.
  \item I haven't had that much direct experience with 
  user-friendly interfaces but I have made what I believe is an easy-to-use
  GUI for my wavelet toolbox for UNIX which is avaliable at 
  {\tt www.imm.dtu.dk/\~{}omni/wt.html} (wavemenu.m).
  On the other hand, I have much experience with efficient data structures.
  One example is an efficient algorithm for dealing with 2d wavelet transforms
  of circulant matrices such as differential operators 
  and various operations on them.
  An algorithm exploiting the particular structure of this problem 
  reduced the storage requirements from quadratic to linear in the 
  number of non-zeros and the algorithmic complexity was reduced similarly.
  This is documented in detail in my PhD dissertation, page $121$ to $162$.   
  \item Efficient code generation is again an intrinsic part of my work.
  Examples are development of vector-parallel wavelet transforms
  for the Fujitsu VPP 300 scientific software library at the ANU.
  The implementation was accompanied by a performance model that
  predicted both sequential and parallel actual performance.
  About 80 \% of peak performance on one processor was achieved,
  and the parallel algorithm was arranged such that the amount of
  communication between any two processors was independent on the problem size
  as well as the number of processors. 
  Finally, it was possible to overlap communication with computations
  so almost linear speedup was obtained. Dr. Markus Hegland can verify 
  the success of this project.
  \item During my work as a consultant at UNI$\bullet$C 
  I did the initial analysis and design of a parallel image analysis
  algorithm for a private company specialising in medical imaging.
  Unfortunately, I had to leave the project before it was finished
  because of my position at the ANU, but I put much effort into 
  transferring the design documents and test programs to 
  my successor and the project has now completed successfully.
\end{enumerate} 

\subsection*{B) Application of knowledge}
 
\begin{enumerate} 
 \item My university degrees (BS, MS, PhD) were all aimed at mathematics, 
 computer science and scientific computing. Since then I have been working
 with either research or actual development in these fields.
 \item I am fluent in several programming languages such as
 Fortran/Fortran90, C/C++, Delphi, and Matlab 
 and experienced in using the operating systems UNIX/Linux and Windows NT. 
 However, I have not combined Fortran and Windows as it is, since my previous 
 work with Windows required a combination of Delphi, C++, MPI, and DCOM.
 I doubt that the Fortran - NT combination could possibly pose any 
 challenges not also encountered with the above mentioned languages.
 \item I am actively involved in code development for Data Mining
 (smoothing, multigrid methods, wavelets) in my current position but 
 also use packages like Mineset occasionally.
\end{enumerate} 

\subsection*{C) Communicaton}
 
\begin{enumerate}
 \item Part of my masters thesis in computer science 
 consisted of writing comprehensive software documentation and
 a full user manual. 
 \item Roskilde University has adobted a very modern education style where
 most activities are project oriented, problem driven and  
 interdisciplinary. It is very hard obtain a degree from 
 Roskilde University if one doesn't have extensive communication 
 skills and the ability to work in groups. 
 %A recent report about this university was made by 
 %Prof. Katherine Legge at La Trope who went there to a 
\end{enumerate} 
    
\subsection*{D) Judgement and E) Adaptability}    

  Making the right decision when faced with many alternatives
  can be very difficult and having done it previously is no guarantee
  that it will happen in a future situation. Having said that,
  I think it is fair to say that getting a post graduate degree
  requires good judgement and quite a bit of adaptability.
  In any case I don't hesitate to listen to the opinions of
  my collegues when faced with a number of competing strategies, 
  nor do I mind offering my opinion to those faced with difficult decisions.
  Regarding the ability to meet deadlines I can give the following example:   
  My 3 year PhD-stipend expired 1 March 1998. This happened to fall
  on a sunday so I handed in my dissertation Monday the 2 March 1998.
  
  My former supervisor Per Christian Hansen, 
  former employer J{\o}rgen Moth and current supervisor
  Dr. Markus Hegland will be happy to verify all these matters.
\end{document}
