\documentclass[12pt,a4paper]{article}

\usepackage{times}

%\textheight=25cm
%\textwidth=16cm 
%\topmargin=-4cm
\oddsidemargin=0cm
\pagestyle{myheadings}
\markright{OMN - 2001\ \hrulefill\ }

%\setcounter{page}{8}

\begin{document}

\section*{Re:SMS210---Response to selection criteria}

\subsection*{Essential}
\begin{itemize}
  \item[1] \textit{Research training}:
  During my studies and subsequent appointments I have worked with
  a wide range of topics from computer science and computational mathematics
  including wavelet methods for solving the 
  non-linear Schr{\"o}dinger equations, an efficient algorithm for the 
  2D wavelet transform of circulant matrices, 
  parallel implementations and performance analyses of wavelet
  transforms and fast Fourier transforms, multi grid techniques, 
  parallel image analysis, image restauration and inverse problems, 
  research in predictive modelling 
  and regression based on wavelets and sparse grids, 
  data mining consultancies, development and application of 
  data exploration software, design and implementation of on-line data 
  analysis on the WEB. 
  I have postdoctoral experience in data mining research and consulting
  from November 1998 to the present.
  \item[2] \textit{Published research}: 
  Publications and research experience are detailed in enclosed CV.  
  Papers enclosed are: 
  \begin{itemize} 
  \item[] \textit{Parallel Performance of Fast Wavelet Transforms} 
  which describes 
  a fully scalable algorithm for computing the 2D wavelet transform. 
  This has relevance in image processing and compression if large 
  images need to be processed quickly (e.g.\ for digital movies). A detailed
  performance analysis reveals why the algorithm is scalable.
  \item[] \textit{High Dimensional Smoothing Based on Multilevel Analysis} 
  describes
  a wavelet based method for substantially reducing the algorithmic 
  complexity of high dimensional surface smoothing. It is also shown 
  that the approximation error introduced is small for smooth functions.  
  \item[] \textit{A Toolbox Approach to Flexible and Efficient Data Mining} 
  describes 
  development of software to facilitate handling of common data exploration 
  tasks very efficiently. This allows for a more rapid data mining process, 
  because most of the time in data mining is normally spent in data 
  exploration.
  \item[] \textit{Wavelet Analysis for Power System Transients.}
  This was an invited tutorial for IEEE Computer Applications in Power
  aimed at non-specialists that I wrote with a 
  colleague (Dr.\ Anthony Galli) from Siemens -- Westing House.
  \end{itemize} \pagebreak
  \item[3] \textit{Research interests and achievements}:
  I wish to pursue activities in research, 
  technology development, and applications as detailed below\label{ref:PE}  
  \begin{itemize}
    \item[] \textbf{Research:} I plan to continue research in data mining
    in general and sparse-grids/wavelet based methods 
    for predictive modelling in particular. The latter is a technique 
    for reducing the complexity of high dimensional surface fitting
    while retaining good accuracy of the computed approximation.
    \item[] \textbf{Technology:} I have played a major role in developing
    software for easy and efficient access to data bases as well as
    integration of some predictive models and clustering. 
    I plan to continue this and extend the integration to include
    other methods such as association rules, decision trees and/or 
    neural networks. 
    In addition, to handle the large data sets and high dimensionality 
    encountered in real-life applications the software developed will
    take advantage of high performance parallel computer architectures.
    \item[] \textbf{Applications:}  
    I have been heavily involved in an ongoing collaboration 
    with Center for Mathematics and Information Science at CSIRO
    in joint consultancies for the Health Insurance Commission (HIC)
    and Department for Health and Age Care (HAC). 
    Currently, I am developing
    technologies for online data exploration to be used by policy makes 
    at the HIC. 
    This is the initial part of a larger three year contract
    which is under preparation and which---if successful---will generate
    profitable interactions outside the ANU.
    I plan to continue and expand this important line of research.
  \end{itemize}
  \item[4] \textit{Degrees}:  
  I hold a PhD in Scientific Computing, an MS in Computer Science,
  and a BS in Mathematics as detailed in the enclosed CV.
  \item[5] \textit{Understanding of Equal Opportunity principles}:
  A direct experience with (lack of) equal 
  opportunities took place during my studies at the NYU in New York City 
  in 1996, where I signed up for the Harlem Tutorial Program organised by 
  International House which is an institution promoting multi cultural 
  and international activities.
  I had the pleasure of tutoring two nine year old boys from Harlem once 
  a week for four months, got to know their family, their church, and their
  life in general.
  While this was a very pleasant experience for all involved, it also 
  became quite clear to me that these boys were inherently 
  disadvantaged for the following reasons: 
  (1) The materials they had access to at school were older and more 
  run-down than those used by schools elsewhere. 
  While not directly inhibiting the learning, this observation gave me
  the impression that less is expected from kids in that area than elsewhere.
  (2) A general problem in the families is the 'lack of fathers' syndrome. 
  Families were almost exclusively held together by mothers or grand-mothers, 
  and few children enjoyed regular communication with their dads.
  Such one-parent families tended to have little surplus to motivate a kid
  to excel at school. 
  (3) The area had a very high unemployment rate and those with jobs would 
  generally be employed in traditional 'blue colour' short-term jobs.
  %4: Even though I didn't notice any direct racial tension, I believe
  %that race still plays a major role in the disadvantage of people in 
  %these areas.
  Clearly, no matter how bright a kid from that area is, he or she 
  will start out with significantly reduced opportunities because
  of less expectations from the society, 
  lack of motivation from home and lack of 'high-achieving'
  role models around him or her.
  Hence, even if a person from this background is admitted to an institution
  under an equal opportunities program, he or she will still be 
  disadvantaged simply because he or she is less likely to be qualified
  for the reasons stated above.
  In conclusion, I think that equal opportunities principles at the 
  university level are an honourable ideal and may be helpful in 
  many situations. However, without true equal opportunities in \emph{all} 
  aspects of society over many generations EO principles are not likely to be 
  as effective as they are claimed to be. 
\end{itemize}

\subsection*{Highly Desirable}

\begin{itemize}
  \item[1] \textit{Leadership and supervision}:
  I participated actively in the supervision of a post graduate 
  summer scholar, Tim Hancock, from December to February this year.
  He was planning to do postgraduate studies in data mining and had 
  contacted our group.
  In addition I have many years of experience in tuition of both 
  under graduates and post graduates during my own studies.
  \item[2--3] \textit{Interaction and Cooperation}: 
  During my work at RSISE I have been active in 
  developing research and consultancy collaborations with colleagues 
  within the ANU such as the ANUSF (Margaret Kahn) or DCS (Peter Christen) 
  as well as outside such as 
  CSIRO (Graham Williams, Simon Hawkins, Rohan Baxter, 
  Michael Fett, Christopher Zoppou), 
  National University of Singapore (Zuowei Shen), 
  Siemens Westinghouse (Anthony Galli)
  or lately Tim Churches from the NSW Department of Health who 
  contacted me in response to my participation at the Ninth International 
  Python conference. He has some funding and we are currently exploring 
  possible collaborations.~\label{ref:Tim}
  \item[4] \textit{Software Development}: 
  I have developed software throughout all of my studies and
  subsequent working experience and I am fluent in a large number of 
  object oriented programming languages including
  Python, C++, Delphi, and Simula. 
  I am also well experienced in 
  systems analysis, extensible and modular design methodologies, 
  robust and efficient implementations, effective debugging, 
  documentation and presentation, as well as application to real-world
  problems and production of end-user programs.
  Examples are a Matlab toolbox for wavelet analysis, 
  Python Toolbox for database exploration with caching, and currently
  design of an online data analysis WEB application.
  \item[5] \textit{Capacity to interact with others}:
  I have successfully interacted with people various fields.
  Examples are Michael Fett (Epidemiology and health science)
  where administrative health data was mined for patterns in the 
  treatment of gastro-intestinal diseases, 
  Christopher Zoppou (Hydrology) where 300 Australian rivers were 
  clustered based on wavelet-energy signatures,
  Magaret Kahn (Astronomy and high performance computing)
  where software for exploration of astronomical time-series was enhanced
  performance-wise through the use of a caching mechanism developed by me,
  Anthony Galli (Engineering) where wavelets were applied in 
  electrical engineering.
\end{itemize}

\subsection*{Desirable}

\begin{itemize} 
  \item[1] \textit{Potential for profitable interactions}: See
  item 3, applications, under \textbf{Essential} on 
  page \pageref{ref:PE} as well as items 2--3 under
  \textbf{Highly Desirable} on page \pageref{ref:Tim}.
  Moreover, I was actively involved in writing the successful APAC 
  proposal for the Data Mining project. 
  \item[2] \textit{Administrative tasks}: 
  I do not have a problem with undertaking a reasonable amount
  of administrative tasks as this is a necessary part of keeping a department
  running.
\end{itemize} 
  
\subsection*{Other}
  I have successfully applied for and been granted permanent
  residency in Australia as a skilled migrant as of 9 March this year. 
  This means that there will be no need for the SMS to sponsor 
  a residency visa should my application be successful. 
\end{document}