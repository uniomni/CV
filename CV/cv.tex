\documentclass[12pt,a4paper]{article}

\usepackage{times}
\pagestyle{myheadings}
\markright{Ole Nielsen - 2006\ \hrulefill\ }


\begin{document}

\section*{Curriculum Vitae -- Ole M{\o}ller Nielsen}

\subsection*{Personal address}
28 Scrivener Street \\
O'Connor ACT 2602 \\
Canberra, Australia \\
+61 2 6262 5209  

\subsection*{Work address}
Geospatial and Earth Monitoring Division (GEMD), 
Risk Assessment Methods Project \\
Geoscience Australia \\
Symonston ACT 2609 \\
Australia \\
+61 2 6249 9048 (direct)\\
+61 2 6249 9986 (fax)\\
Email: Ole.Nielsen@ga.gov.au \\
URL: http://datamining.anu.edu.au/\verb+~+ole
 
% \subsection*{Work address}
% Centre for Mathematics and its Applications \\
% School of Mathematical Sciences \\           
% Australian National University \\   
% Canberra ACT 0200
% Australia \\
% +61 2 6125 3873 (direct)\\
% +61 2 6125 5549 (fax)\\
% +61 2 6125 2897 (secr.) \\
% Email: Ole.Nielsen@anu.edu.au \\
% URL: http://datamining.anu.edu.au/\~{}ole
 
\subsection*{Education}
\begin{itemize}
\item Doctor of Philosophy (May 1998) \\
{\bf Technical University of Denmark} \\ 
Department of Mathematical Modelling  \\
Thesis: ``Wavelets in Scientific Computing''\\
%Available at {\tt http://datamining.anu.edu.au/\~{}ole} \\ 
Grade Point Average: 11.0/13\\    %All 11's + one A
URL: http://datamining.anu.edu.au/\verb+~+ole/work/publications/thesis.pdf

\item  Master of Science (November 1993) \\
{\bf Roskilde University, Denmark} \\
Department of Computer Science \\ 
Thesis: ``DISCO -- DIScrete and COntinuous simulation''\\ 
Grade Point Average: 11.4/13  %(10.5 + 12 + 11 + 12)/4

\item  Batchelor of Science (June 1990) \\
{\bf Roskilde University, Denmark} \\
Department of Mathematics \\ 
Grade Point Average: 11.5/13  %(10+13)/2
\end{itemize}

\newpage

\subsection*{Professional history}
\input{cv_exp.sub}


\subsection*{My mission}
To involve myself in research and development of  
technical or scientific technologies aimed at
concrete applications or consultancies 
that have a high degree of practical impact to society in 
an area outside the immediate realms of science. 
%that have a high degree of relevance to society. 


%outside the academia.
%Relevance to society is very high on my list of priorities.

%----------
%I am currently employed in a Post Doc position at the ANU
%as a researcher in Wavelets, Datamining, and 
%Parallel Computing.
%However, this is a temporary position, and my ambitions
%are ultimately to do scientific software development
%aimed at concrete products presumably in industry.
%
%My expertises are mathematical modelling, 
%signal processing, and computer science.
%Aside from an extensive variety of technical skills 
%I think of myself as a reliable teamplayer with 
%very good communication skills.

\subsection*{General expertise}
\begin{itemize} 
  \item Automation of complex processes.
  \item Experience in balancing time, quality, scope and resources to achieve tangible outcomes.
  \item Expertise in managing development of scientific software. 
  \item Strategic thinking, team play, clear communication and leadership.
  \item Able to overcome problems to get 'the job done'.  
\end{itemize}


\subsection*{Selected Math skills}
\begin{itemize} 
  \item Numerical Linear Algebra
  \item Computational Fluid Mechanics (Finite-\{volumes, elements, differences\} discretisations, 
  Non-linear differential equations such as the Shallow Water Wave Equations and the 
  Nonlinear Schr\o{}dinger equation).  
  \item Data Mining and predictive modelling (Classification and Regression Trees, High dimensional 
  surface fitting, Machine learning, Penalised Least-squares techniques).     
  \item Wavelets, Fourier transforms and filter banks.
  \item Linear Programming.
\end{itemize}


\subsection*{Selected IT skills}
\begin{itemize} 
  \item Programming languages: Python, C/C++, Matlab, Fortran 77/90, Delphi
  \item Distributed computing tools: MPI, DCOM, Fujitsu VP-Fortran
  \item WEB programming: HTML, Javascript, CGI, PHP, WebWare, MapServer 
  \item Data management: XML, MySQL, Excel 
  \item Software developing tools: CVS, Subversion, TRAC, unit testing, make
  \item Operating systems: Debian Linux, Windows NT, Solaris
  \item Text processing: Emacs, \LaTeX, Xfig, MS-Word, Open Office
\end{itemize}


\section*{Some achievements}
\begin{itemize} 
  \item A new modelling capability that has enabled Geoscience Australia to simulate 
  impacts of tsunami or storm surge disasters on the built environment and to present the 
  results in forms that are easily interpreted. The software developed is a 
  state-of-the-art hydrodynamic modelling tool that has raised the bar for what is 
  technically and conceptually possible and has created significant interest from 
  both academia and industry. In particular, we are able to predict what \emph{consequences} a 
  hydrological disaster may have on a particular community rather than a simple statement about 
  waveheights off-shore. The work of me and my team was awarded the "EMA Safer Communities Award 2005" and 
  is the main reason Emergency Management Australia is now recognising the benefits of entering a 
  formal collaboration with science to achieve their outcomes.
  \item A corporate high performance computing capability in my organisation. 
  This involved building a prototype parallel Linux cluster, developing and selling the business case to the 
  senior management team, setting up a corporate wide special interest group, putting in the capital bid, 
  procuring the hardware through a formal tender process, leading the acceptance testing and finally developing and deploying parallel software as well as coaching colleagues in using the cluster.
  \item Influencing my workplace to take up modern software development methodologies and practices that 
  have improved the quality, speed and audit trail of corporate software.
  \item A binding for the Message Passing Interface (MPI) for the Python programming language.
  This binding was publishe as open source and is currently
  used widely in the scientific community (bio-informatics, health and datamining and modelling). 
  PyPar is available at http://datamining.anu.edu.au/pypar
  \item The development of a WEB-enabled 
  data exploration tool online analysis of large Health care databases 
  at the Health Insurance Commission. This involved datamining and record linkage of about 80 million
  MBS claims and 50 million PBS claims and required the development of special purpose software to 
  handle these volumes. 
  \item Designed and developed an open software library for data mining
  of large relational databases using the scripting language Python
  This library is successfully used for fast, interactive access to large 
  databases at the Australian National University (ANU) and 
  Commonwealth Science and Industrial Research Organisation (CSIRO).
  \item Developed database of currently 0.5 million 
  GPS coordinates searchable by proximity to a given location 
  and by optional keywords. The search engine is fast
  due to data base tables dynamically organised in a tree structure. 
  Wrote WEB front end for searching, retrieving and uploading GPS waypoints
  using the GPS search engine.
  \item Successfully implemented and applied a wavelet based algorithm 
  for surface fitting of high-dimensional data for use in predictive modelling.
  \item Developed vector-parallel fast wavelet transforms
  for the Fujitsu scientific software library.
  % at the Australian National University.
  The implementation was accompanied by a performance model that
  predicted both sequential and parallel actual performance.
  About 80 \% of peak performance on one processor was achieved,
  and the parallel efficiency was {\em independent} of the problem size
  as well as the number of processors. 
  \item Developed an efficient algorithm for 
  wavelet transforms of circulant matrices 
  and various operations on them.
  Exploiting the particular structure of this problem 
  reduced the storage requirements as well as the algorithmic complexity 
  from {\em quadratic} to {\em linear} in the number of non-zeros.
  \item Did the initial analysis, design, and architecture evaluations 
  towards the parallelisation of a medical image analysis algorithm.
  \item Published an easy-to-understand wavelet tutorial
  in IEEE Computer Applications in Power, Volume 12, Number 1, January 1999.
  \item Have given technical and scientific presentations at 
  various institutions and conferences including 
  International Disaster Reduction Conference (Switzerland 2006); 
  UNESCAP High Level Technical meeting on Tsunami Disasters (Thailand 2005); 
  Australian Earth Quake Engineers annual meeting (GA 2006), Dynamic Earth Conference (ANU 2006), 
  Computational Techniques and Applications Conferences (CTAC 1999, 2006); 
  Modelling and Simulations (MODSIM 2005), Ninth Internal Python Conference (Californa 2001), 
  Australian Epidemiological Association (Sydney); NSW Health; IBM Yorktown heights (USA); 
  St Catherine's College, Oxford (UK); Umeaa University (Sweden); 
  UNI$\bullet$C (Denmark); and many more.
\end{itemize} 



%\subsection*{Ph.D.\ dissertation}
%\input{cv_phd.sub}
%
%\subsection*{Lectures and presentations given}
%\input{cv_pres.sub}

%\subsection*{Other conferences and meetings attended}
%\input{cv_meet.sub}
%
%\clearpage
\subsection*{Selected publications}
%\subsection*{Publications}
\input{cv_publ.sub}

%\pagebreak
%\subsection*{Memberships, Awards and Grants}
%\input{cv_grants.sub}

%\subsection*{Services to the Scientific Community}
%\input{cv_serv.sub}


\newpage
\subsection*{References}

\noindent Dr. Matthew Hayne, (Matthew.Hayne@ga.gov.au)\newline
Risk Assesment Methods Project  \newline
Geoscience Australia \newline
Phone: +61 2 6149 9536 \newline
Fax:\ \ \ \ \   +61 2 6149 9999 \newline

\noindent Dr. Peter Christen, (Peter.Christen@anu.edu.au)\newline
Department of Computer Science  \newline
Australian National University \newline
Phone: +61 2 6125 5690 \newline
Fax:\ \ \ \ \   +61 2 1349 2149 \newline

%\noindent Dr. Graham Williams, (Graham.Williams@cmis.csiro.au)\newline
%CSIRO Mathematical and Information Sciences  \newline
%%Building 108 North Road, 
%Australian National University \newline
%Phone: +61 2 6216 7042 \newline
%Fax:\ \ \ \ \   +61 2 6216 7111 \newline

\noindent Dr Stephen Roberts, (Stephen.Roberts@anu.edu.au)\newline
Mathematical Sciences Institute \newline
Australian National University \newline
Canberra ACT 0200, Australia \newline
Phone: +61 2 6125 8634 \newline
Fax:\ \ \ \ \  +61 2 6125 8645 \newline


\noindent Dr Markus Hegland, (Markus.Hegland@anu.edu.au)\newline
Mathematical Sciences Institute \newline           
%%John Dedman Mathematical Sciences Building \#27 \newline
Australian National University \newline   
Canberra ACT 0200, Australia \newline    
Phone: +61 2 6125 4501 \newline         
Fax: \ \ \ \ +61 2 6125 5549 \newline

\noindent Senior Consultant J{\o}rgen Moth, (Jorgen.Moth@uni-c.dk)\newline
UNI$\bullet$C (The Danish Super Computing Centre) \\
%Technical University of Denmark, Building 304 \newline
2800 Lyngby, Denmark \newline
Phone: +45 35 87 88 89 \newline
Fax:\ \ \ \ \  +45 35 87 89 90 \newline


%\noindent Professor Vincent Allan Barker, (vab@imm.dtu.dk)\newline
%Department of Mathematical Modelling \newline
%Technical University of Denmark, Building 305 \newline
%2800 Lyngby, Denmark \newline
%Phone: +45 4525 3074 \newline
%Fax:\ \ \ \ \  +45 4593 2373\newline

\noindent Professor Per Christian Hansen, (pch@imm.dtu.dk)\newline
Department of Mathematical Modelling \newline
Technical University of Denmark, Building 305 \newline
2800 Lyngby, Denmark \newline
Phone: +45 4525 3097 \newline
Fax:\ \ \ \ \  +45 4593 2373\newline


%\noindent Professor Stig Andur Pedersen, (sap@ruc.dk)\newline
%Department of Mathematics\newline
%Roskilde University\newline
%%Box 260\newline
%4000 Roskilde, Denmark\newline
%Phone: +45 4674 2265\newline
%Fax:\ \ \ \ \  +45 4674 3020\newline

%\noindent Dr. Lionel Watkins, (l.watkins@auckland.ac.nz)\newline
%Department of Physics\newline
%University of Auckland\newline
%%Private Bag 92019\newline
%Auckland, New Zealand\newline
%Phone: +64 9 373 7599 - 8878\newline
%Fax:\ \ \ \ \  +64 9 373 7445\newline

%\noindent Associate Professor Johnny Ottesen, (johnny@mmf.ruc.dk)\\
%Department of Mathematics\\
%Roskilde University\\
%4000 Roskilde, Denmark\\
%Phone: +45 4674 2298\\
%Fax:   +45 4674 3020\\

\end{document}

