\documentclass[11pt,a4paper]{article}

\usepackage{times}
\usepackage{url}

\pagestyle{empty}

\textheight=25cm
\textwidth=15cm
\topmargin=-2cm
\oddsidemargin=0cm

\begin{document}

\begin{center}
  Ole Nielsen \\
  Jalan Patra Kuningan X, No 3\\
  Jakarta Selatan 12950, Indonesia \\
  P: +62 811 820 4637,\ \ \ E: nielso@ausaid.gov.au
\end{center}

\begin{center}
  \hrulefill \\
  {\bf Mission} \\[-0.2cm]
  \hrulefill
\end{center}
\emph{Applying science and technology to problems that matter.}
\begin{center}
  \hrulefill \\
  {\bf Career History} \\[-0.2cm]
  \hrulefill
\end{center}

\begin{itemize}
\item {\em Mar 2010 -- Mar 2013}: Numerical Modeller, Australia-Indonesia Facility for Disaster Reduction, AusAID, Indonesia.
      Oversight of software engineering and computational infrastructure supporting the Indonesian government disaster management agency in better planning and decision making.
\item {\em Mar 2003 -- Mar 2010}: Senior Computational Scientist, Geoscience Australia.
      Research, development and application of natural hazard models.
\item {\em Sep 2003 -- Dec 2003}: Visiting Professor,
      Department of Mathematics,
      Suranaree University of Technology, Nakhorn Ratchasima, Thailand. Teaching PhD course in
      High Performance Computing.
\item {\em Nov 1998 -- Feb 2003}: Research Fellow,
      School of Mathematical Sciences, Australian National University.
      Research in enterprise datamining.
\item {\em Mar 1998 -- Oct 1998}: Scientific Computing Consultant \\
      UNI$\bullet$C, Danish Computing Centre for Research and Education.\\
      Design and development of parallel image analysis algorithm.
\end{itemize}



\begin{center}
  \hrulefill \\
  {\bf Education} \\[-0.2cm]
  \hrulefill
\end{center}

\begin{itemize}
\item Doctor of Philosophy (May 1998) \\
{\bf Technical University of Denmark} \\
Department of Mathematical Modelling  \\
Thesis: ``Wavelets in Scientific Computing''\\
%Available at {\tt http://datamining.anu.edu.au/\~{}ole} \\
%Grade Point Average: 11.0/13    %All 11's + one A

\item  Master of Science (November 1993) \\
{\bf Roskilde University, Denmark} \\
Department of Computer Science \\
Thesis: ``DISCO -- DIScrete and COntinuous simulation''\\
%Grade Point Average: 11.4/13  %(10.5 + 12 + 11 + 12)/4

\item  Batchelor of Science (June 1990) \\
{\bf Roskilde University, Denmark} \\
Department of Mathematics \\
%Grade Point Average: 11.5/13  %(10+13)/2
\end{itemize}


\pagebreak

\begin{center}
  \hrulefill \\
  {\bf Main achievements} \\[-0.2cm]
  \hrulefill
\end{center}

\begin{itemize}
  \item A novel and user friendly application for rapid and reproducible estimation of impact from natural disasters called InaSAFE which is available at \url{www.inasafe.org}.
  Version 1.0 was publicly and officially launched at the Asian Ministerial Conference on Disaster Risk Reduction in October 2012 and shown to the president of Indonesia as the centrepiece of Australian Indonesian cooperation in Disaster Management. InaSAFE is currently being provided to local and national disaster managers in Indonesia. The World Bank who is a partner in this project is building web applications based on InaSAFE for use world wide.
  \item A new modelling capability that can simulate
  impacts of tsunami, storm surge or flood disasters on the built environment.
  This work was awarded the Emergency Management Australia "Safer Communities Award" in 2005 and 2007 as well as the
  Asia-Pacific Spatial Excellence Award in 2007.
  The software underpinning this work is available at \url{http://sourceforge.net/projects/anuga}
  \item A corporate high performance computing capability in my organisation.
  This involved building a prototype parallel Linux cluster, developing and presenting the business case to the
  senior management team, setting up a corporate wide special interest group, managing the installation process
  from tender to final acceptance testing.
  \item Influencing my workplace to take up modern software development methodologies and practices that
  have improved the quality, speed and audit trail of corporate software.
  \item A binding for the Message Passing Interface (MPI) for the Python programming language.
  Pypar is open source, used widely in the scientific community (bio-informatics, health and datamining and modelling) and underpins projects such as ANUGA, python-FALL3D, URS-TSUNAMI and EQRM.
  Pypar is available at \url{http://sourceforge.net/projects/pypar}
  \item The development of a WEB-enabled
  data exploration tool online analysis of large Health care databases
  at the Health Insurance Commission. This involved datamining and record linkage of about 80 million
  MBS claims and 50 million PBS claims and required the development of special purpose software to
  handle these volumes.
  %\item Designed and developed an open software library for data mining
  %of large relational databases using the scripting language Python
  %This library is successfully used for fast, interactive access to large
  %databases at the Australian National University (ANU) and
  %Commonwealth Science and Industrial Research Organisation (CSIRO).
  %\item Developed database of currently 0.5 million
  %GPS coordinates searchable by proximity to a given location
  %and by optional keywords. The search engine is fast
  %due to data base tables dynamically organised in a tree structure.
  %Wrote WEB front end for searching, retrieving and uploading GPS waypoints
  %using the GPS search engine.
  %\item Successfully implemented and applied a wavelet based algorithm
  %for surface fitting of high-dimensional data for use in predictive modelling.
  \item Developed vector-parallel fast wavelet transforms
  for the Fujitsu scientific software library.
  % at the Australian National University.
  The implementation was accompanied by a performance model that
  predicted both sequential and parallel actual performance.
  About 80 \% of peak performance on one processor was achieved,
  and the parallel efficiency was {\em independent} of the problem size
  as well as the number of processors.
  \item Developed an efficient algorithm for
  wavelet transforms of circulant matrices
  and various operations on them.
  Exploiting the particular structure of this problem
  reduced the storage requirements as well as the algorithmic complexity
  from {\em quadratic} to {\em linear} in the number of non-zeros.
  %\item Did the initial analysis, design, and architecture evaluations
  %towards the parallelisation of a medical image analysis algorithm.
  %\item Published an easy-to-understand wavelet tutorial
  %in IEEE Computer Applications in Power, Volume 12, Number 1, January 1999.
  %\item Have given technical and scientific presentations at
  %various institutions and conferences including
  %International Disaster Reduction Conference (Switzerland 2006);
  %UNESCAP High Level Technical meeting on Tsunami Disasters (Thailand 2005);
  %Australian Earth Quake Engineers annual meeting (GA 2006), Dynamic Earth Conference (ANU 2006),
  %Computational Techniques and Applications Conferences (CTAC 1999, 2006);
  %Modelling and Simulations (MODSIM 2005), Ninth Internal Python Conference (Californa 2001),
  %Australian Epidemiological Association (Sydney); NSW Health; IBM Yorktown heights (USA);
  %St Catherine's College, Oxford (UK); Umeaa University (Sweden);
  %UNI$\bullet$C (Denmark); and many more.
\end{itemize}


\begin{center}
  \hrulefill \\
  {\bf Selected publications} \\[-0.2cm]
  \hrulefill
\end{center}

\input{cv_publ.sub}



\end{document}

